% Template:     Informe LaTeX
% Documento:    Archivo principal
% Versión:      8.2.3 (03/04/2023)
% Codificación: UTF-8
%
% Autor: Pablo Pizarro R.
%        pablo@ppizarror.com
%
% Manual template: [https://latex.ppizarror.com/informe]
% Licencia MIT:    [https://opensource.org/licenses/MIT]

% CREACIÓN DEL DOCUMENTO
\documentclass[
	spanish, % Idioma: spanish, english, etc.
	letterpaper, oneside
]{article}

% INFORMACIÓN DEL DOCUMENTO
\def\documenttitle {Martín E. Bravo, Franco González y Felipe Avendaño A.} 
\def\documentsubtitle {}
\def\documentsubject {Tema a tratar}

\def\documentauthor {Nombre del autor}
\def\coursename {Matemáticas Discretas}
\def\coursecode {CC3101-1}

\def\universityname {Universidad de Chile}
\def\universityfaculty {Facultad de Ciencias Físicas y Matemáticas}
\def\universitydepartment {Departamento de la Universidad}
\def\universitydepartmentimage {departamentos/dcc}
\def\universitydepartmentimagecfg {height=1.57cm}
\def\universitylocation {Santiago de Chile}

% INTEGRANTES, PROFESORES Y FECHAS
\def\authortable {
	\begin{tabular}{ll}
		Integrantes:
		& \begin{tabular}[t]{l}
			Integrante 1 \\
			Integrante 2
		\end{tabular} \\
		Profesor:
		& \begin{tabular}[t]{l}
			Profesor 1
		\end{tabular} \\
		Auxiliar:
		& \begin{tabular}[t]{l}
			Auxiliar 1
		\end{tabular} \\
		Ayudantes:
		& \begin{tabular}[t]{l}
			Ayudante 1 \\
			Ayudante 2
		\end{tabular} \\
		\multicolumn{2}{l}{Fecha: \today} \\
		\multicolumn{2}{l}{\universitylocation}
	\end{tabular}
}

% IMPORTACIÓN DEL TEMPLATE
\input{template}

% INICIO DE PÁGINAS
\begin{document}

% PORTADA
\templatePortrait

% CONFIGURACIÓN DE PÁGINA Y ENCABEZADOS
\templatePagecfg

% CONFIGURACIONES FINALES
\templateFinalcfg

% ======================= INICIO DEL DOCUMENTO =======================

\section*{Pregunta 1}
\subsection*{1.}
Se tiene un  algoritmo $A$ de caja negra de resolución SAT, es decir, un dispositivo que toma una fórmula de lógica proposicional $\phi$, A($\phi$) es verdadero si $\phi$ es satisfacible.


a) Algoritmo que utiliza $A$ como subrutina para determinar si $\phi$ es una tautología:

\begin{lstlisting}[language=Python]
def esTautologia(p):
    if A(~p) == 0: #Si ~p es insatisfacible
        return True
    else:
        return False
\end{lstlisting}
Probaremos si el algoritmo anterior es correcto:
\begin{proof}

    PDQ: $\neg \phi$ es insatisfactible $\Leftrightarrow \phi$ es tautología
    
    $\neg \phi$ es insatisfactible $\Leftrightarrow (\emptyset \cup \{\neg \phi\})$ es insatisfactible
    
    $\hspace{33mm} \Leftrightarrow \emptyset \models \phi$

    $\hspace{33mm} \Leftrightarrow \sigma(\phi) = 1$

    $\hspace{33mm} \Leftrightarrow \phi$ es Tautología
\end{proof}

b) Se tienen 2 fórmulas proposicionales $\phi$ y $\varphi$, se va a determinar si tienen los mismos valores de verdad. Algoritmo que utiliza $A$ para responder a esta pregunta:

\begin{lstlisting}[language=Python]
def equivalentes(p, q):
    if A(~((~p or q) and (p or ~q))) == 0:
        return True
    else:
        return False
\end{lstlisting}

Probaremos si el algoritmo anterior es correcto:

\begin{proof}
    PDQ: $\neg ((\neg \phi \vee \varphi)\land(\phi \vee \neg \varphi))$ es insatisfactible $\Leftrightarrow \phi \equiv \varphi$

    $\neg ((\neg \phi \vee \varphi)\land(\phi \vee \neg \varphi))$ es insatisfactible $\Leftrightarrow (\neg \phi \vee \varphi)\land(\phi \vee \neg \varphi)$ es Tautología

    $\hspace{67mm} \Leftrightarrow (\phi \Rightarrow \varphi) \land (\varphi \Rightarrow \phi)$ es Tautología

    $\hspace{67mm} \Leftrightarrow (\phi \Leftrightarrow \varphi)$ es Tautología

    $\hspace{67mm} \Leftrightarrow \sigma(\phi) = \sigma(\varphi)$

    $\hspace{67mm} \Leftrightarrow \phi \equiv \varphi$
    
\end{proof}

\newpage

c) Se tiene una fórmula proposicional $\phi$ con $n$ variables que se sabe que es satisfacible. Algoritmo que utiliza $A$ como subrutina para obtener una asignación satisfactoria para $\phi$ utilizando como máximo $n$ llamadas a $A$:

Sean $\phi_1,...,\phi_n$ proposiciones de la fórmula, luego sea $\phi '$ tal que $\sigma(\phi_1)=1$\\

Si $A(\phi')$ satisfacible $\Rightarrow \sigma(\phi_1)=1$

Si $A(\phi')$ insatisfacible $\Rightarrow \sigma(\phi_1)=0$\\

Luego $\phi''$ tal que $\sigma(\phi_1) =$ \{valor con el que es satisfacible\} y $\sigma(\phi_2)=1$, se repite el proceso anterior n veces hasta encontrar todas las valuaciones $\sigma (\phi_k)$, $k\in {1,...,n}$.\\
\begin{proof}
	Para probar la correctitud se procederá por el principio de inducción:\\
	\fbox{Caso Base:}\\
Sea $\phi$ una formula lógica con $\phi_{1}$ su única proposición y se sabe que existe $\sigma(\phi_1) $ tal que $\sigma(\phi) = 1$. Sea entonces $\phi'$ tal que $\sigma(\phi_1) = 1$, como la fórmula depende solo de $\phi_1$, si con $\sigma(\phi) = 1$ se cumple que $A(\phi')$ entrega satisfacible también se cumple que $\sigma(\phi')=1$, por el contrario si $A(\phi')$ es insatisfacible entonces basta con tomar $\sigma(\phi) = 0$, con esa valuación se tendría el otro caso.\\
	\fbox{Caso Inductivo:}\\
Supongamos que existen n-1 valuaciones ${\sigma(\phi_1),...,\sigma(\phi_{n-1})}$ con las que se cumple $A(\phi^{(n-1)})$ entrega satisfacible. Como se sabe que $\phi^{(n-1)}$ es satisfacible entonces debe existir una valuación $\sigma(\phi_n)$ tal que $\sigma(\phi^{(n-1)})=1$. Entonces para encontrar la n-ésima valuación basta con tomar $\phi^{(n)}$ tal que $\sigma(\phi_n)=1$, si con esa valuación $A(\phi^{(n)})$ entrega que es satisfacible entonces encontramos todas las valuaciones con las que $\sigma(\phi)=1$. En caso de que $A(\phi_{(n)})$ entregue insatisfacible, basta tomar $\sigma(\phi_n)=0$.\\
\end{proof}
El algoritmo anterior nos entrega las n valuaciones de las variables de la formula proposicional con solo n llamadas.\\
\newpage

\subsection*{2.}
Notemos que $\sigma(a_i) = \sigma(b_i)$ es un factor importante al momento de sumar los números binarios, por lo que necesitamos definir una función auxiliar que indique si esto se cumple o no. Esto se consigue directamente de reescribir y manipular $a_i$ XOR $b_i$:\\
$$iguales_i = \neg(\neg a_i \vee \neg b_i) \vee \neg(a_i\vee b_i) $$\\
Donde $\sigma(iguales_i) = 1$ si y solo si $\sigma(a_i) = \sigma(b_i)$.\\

Otro factor importante es cuando ocurre lo que denominamos 'arrastre', lo cual ocurre solo de las siguentes formas para este caso:
($1 + 1 = 0$ con arrastre) y ($1 + 1 + 1 = 1$ con arrastre). Lo que hace el arrastre, en esencia, es que si ocurren una de las sumas anteriormente descritas entonces a la siguiente posición se la suma un $1$ extra. Definimos lógicamente si es que se 'arrastró' o no un $1$ desde la posición i-1 hasta la posición i como la función auxiliar siguente:

$$arrastre_i =  \neg ( \phi_{i-1} \vee iguales _{i-1} ) \vee \neg(\neg a_{i-1} \vee \neg b_{i-1}) $$


Donde $\sigma(arrastre_i) = 1$ si y solo si se arrastró un $1$ desde la posición i-1 hacia la posición i. 

\vspace{5mm}

Dadas las definiciones anteriores, construimos nuestras fórmulas $\phi_i$, distinguendo los siguentes casos:\\

\fbox{Caso inicial} (i=0) Para este caso, $\sigma(\phi_i) = 1$ ssi $\sigma(a_i) \neq \sigma(b_i)$

$$\Rightarrow \phi_i = \neg iguales_i \textit{ , donde } i=0$$

\fbox{Caso intermedio} ($i\in \{1,...,n-1\}$) (Dado $n\geq2$) Necesitamos que $\sigma(\phi_i) = 1$ ssi  $\sigma(a_i) \neq \sigma(b_i)$ y que no haya ocurrido algun arrastre hacia la posición i, o que sí haya ocurrido arrastre hacia la posición i, pero que se cumpla que $\sigma(a_i) = \sigma(b_i)$. Es decir:

$$\phi_i = \neg(iguales_i \vee arrastre_i) \vee \neg( \neg arrastre_i  \vee \neg iguales_i ) \textit{ , donde } i\in \{1,...,n-1\}$$

\fbox{Caso final} (i=n): No existen $a_n$ ni $b_n$ para compararlos entre ellos, y notamos que $\phi_{i=n}$ depende unicamente del arrastre que haya ocurrido desde la posición anterior, de la siguiente forma:

$$\phi_i = arrastre_i \textit{ , donde } i=n$$

Con esto se han construido, para todo $i \in \{0,1,...,n-1,n\}$, los $\phi_i$ tales que: $\sigma(\phi_i) = 1$ ssi el i-ésimo bit de la suma es un 1. Notemos que las construcciones fueron expresadas solo a base de los operadores $\{\neg,\vee\}$ (incluyendo las funciones auxiliares, las cuales también fueron definidas solo a base de los operadores indicados), con lo que se cumple con la restricción dada.




\newpage
\section*{Pregunta 2}

Considerando la estructura lógica relacional como $A=(\mathbb{N};+;\times)$, se tiene como dominio los naturales y como relaciones ternarias la adición y la multiplicación, se formalizarán los siguientes predicados:

1. El conjunto de todos los triples $(m,n,p)$ tal que $n\neq 0$ y $p = \lfloor \frac{m}{n}\rfloor$
Se define:
$$Cero(x):=\forall y, \times(x,y,x) \land +(x,y,y)$$
$$a<b := \exists k, +(a,k,b)\land \neg Cero(k)$$

Luego:

$$triples(m,n,p):=\neg Cero(n) \land (\exists k, \exists r, r<n, \times(n,p,k) \land +(k,r,m))$$

2. El conjunto de los triples $(m,n,p)$ tal que m $\geq$ n y $m \equiv n \mod p$, es decir, $m-n$ es divisible por $p$.
Se define:
$$a\geq b := \exists k, +(b,k,a)$$    %   k pertenece a los naturales (>=0)
 $$a|b := \exists x,(\times(a,x,b)\land \neg Cero(a))$$
 Luego:
 $$triples(m,n,p):= \exists k, m\geq n \land +(k,n,m) \land p|k$$


 3. El conjunto de todos los $n$ que son de la forma $2^p$, para algún $p\geq 0$
 $$Uno(x):=\forall y, \times(x,y,y)$$
 $$Dos(x):=\exists y, Uno(y) \land +(y,y,x)$$
 $$2^p(n)= \exists x, Dos(x) \land [Uno(n)\vee (\exists k,\neg Cero(k) \land \times(k,x,n)\land 2^p(k))]$$

 4. El conjunto de todos los pares $(p,q)$ de números primos gemelos. Donde los números primos $(p,q)$ son números primos gemelos si, siendo $q>p$, se cumple $q-p=2$:
Se define:
$$x \neq y := x < y \vee y < x$$
 $$Primo(x):=\forall z,(Uno(z) \Rightarrow z|x ) \wedge \forall z, (\neg Uno(z)\wedge z\neq x \Rightarrow \neg z|x ) \wedge (x|x \wedge \neg Uno(x))$$
Luego:
 $$primosGemelos(p,q) := \exists k, Dos(k) \land [Primo(p) \land Primo(q)] \land [p < q \land +(k,p,q)]$$



 5. El conjunto de los números racionales. Un número racional significa que $p$ y $q$ no deben tener factores en común distintos de $1$ y $-1$:
Se define:
$$MenosUno(x) = \exists k,\exists c, Uno(k) \land Cero(c) \land +(k,x,c)$$
Luego:
$$Racionales(p, q) := \neg Cero(q) \land \forall x([\;x|p \land x|q\;] \Rightarrow [Uno(x)\vee 
 MenosUno(x)])$$


\newpage
\section*{Pregunta 3}

1. Se espera que todas las listas sean de largo mayor > 0. En el caso de que la lista tenga 1 elemento solo tenemos una opción para elegir, si se tienen 2 elementos, como buscamos los elementos disjuntos, elegimos el mayor, si la lista tiene largo mayor a 2 seleccionamos la mayor suma entre la lista sin el ultimo elemento, la lista con el ultimo elemento y la lista que solo contiene al ultimo elemento:

\begin{equation}
    \label{}
    F(n) = \left\{
          \begin{array}{ll}
            0 & \mathrm{si\ } n = 0 \\
            \max(a_1,0) & \mathrm{si\ } n = 1 \\
            \max(F(n-1),\; F(n-2)+a_n) & \mathrm{si\ } n > 2
          \end{array}
        \right.
\end{equation}

\begin{proof}
	Para probar la correctitud se procederá por el principio de inducción:

	\fbox{Caso Base:}

    $n=0$
    $$F(0) = 0$$
    En el caso de que se entregue una lista vacía, la máxima satisfacción posible es 0, por lo que $F(0)$ debe devolver 0.

	$n=1$
    $$F(1) = \max(a_1, 0)$$
    Como la lista solo contiene un valor solo existen 2 sublistas posibles, la lista $[a_n]$ o la lista vacía. Si la lista $[a_n]$ es mayor o igual a 0 entonces se retorna su único valor, en caso contrario se entrega el vacío, en ambos casos se obtiene la sublista con la máxima satisfacción.

	\fbox{Caso Inductivo:}

    PDQ: $\forall n:  F(0) \land F(1) \land \cdots \land F(n-1) \land F(n) \Rightarrow F(n+1)$
    
    $\Rightarrow$ Debemos mostrar que el algoritmo calcula correctamente $F(n+1)$. Cuando el algoritmo calcula $F(n+1)$, este establece que:

    $$F(n+1) = \max(F(n), F(n-1) + a_{n+1}\;,a_{n+1})$$

    Si se tiene una lista de largo n $[a_1,...,a_n]$ y le agregamos un elemento $a_{n+1}$, tenemos que F(n) y F(n-1) existen y nos devuelven los respectivos máximos, si queremos agregar a $a_{n+1}$ a las opciones tendremos que sumárselo a F(n-1), de esta forma nos aseguramos que la sublista que se genera solo contiene elementos no adyacentes. En el caso de que agregar el elemento $a_{n+1}$ y quedarnos con la máxima sublista en el conjunto $[a_1,...,a_{n-1}]$ no nos entregue la máxima suma también se tiene la opción de que la máxima suma sea F(n). Al buscar el máximo entre F(n) y F(n-1)+$a_{n+1}$ nos aseguramos que obtendremos la máxima satisfacción posible de todas las sublistas y que ninguna sublista contiene elementos adyacentes a otros en la lista original.
    

\end{proof}
\newpage

2. Algoritmo que calcula $F$ de forma recursiva:

\begin{lstlisting}[language=Python]
    def F(n, lista): #n = largo
        if n == 0:
            return 0
        else if n == 1:
            return max(a[0],0)
        else:
            return max(F(n - 1), F(n - 2) + a[n-1])
\end{lstlisting}

La complejidad de este algoritmo es $\Theta(2^n)$, como en cada llamada se vuelve a llamar otras 2 veces entonces por cada llamada duplica el tamaño del problema.\\


3. Algoritmo que calcula $F$ de forma iterativa

\begin{lstlisting}[language=Python]
    def F(n, lista):
        Fn_1 = 0
        Fn_2 = 0
        for i in range(n):
            ( Fn_1, Fn_2 ) = ( max(Fn_1, Fn_2+lista[i]),  Fn_1 )

        return Fn_1

\end{lstlisting}
El algoritmos anterior tiene complejidad $\Theta(n)$, ya que su complejidad esta en el loop, que pasa por todos los elementos de la lista original.

% FIN DEL DOCUMENTO
\end{document}